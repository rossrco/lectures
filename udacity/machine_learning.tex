\documentclass{article}
\usepackage{amsmath}
\usepackage{mathtools}
\usepackage{framed} %for important notes that need to be framed
\usepackage{tikz}
\usepackage{multicol} %for multi-column bullet points
\usepackage{float} %for placing figures at the exact place as in the code
\usepackage{lscape} %for inserting landscape pages
\usepackage{makecell} %for specifying line breaks within table cells

\usepackage[a4paper, total={7in, 10in}]{geometry} %for formatting the sheet size

\usepackage{graphicx}
\graphicspath{ {./visualizations/} } %for inserting figures

\usepackage{hyperref} %for hyperlinks
\hypersetup{
    colorlinks = true,
    linkcolor = blue,
    filecolor = magenta,
    urlcolor = cyan,
}

\title{Udacity Machine Learning Nanodegree - Course Notes}

\author{Ruslan Kozhuharov}

\begin{document}
\pagenumbering{gobble}
\maketitle
\tableofcontents
\newpage
\pagenumbering{arabic}
\setlength{\parskip}{1em}

\section{Model Evaluation}
\subsection{Confusion Matrix}
The following table is called a confusion matrix:

\begin{center}
\begin{tabular}{ |l|l|l| }
  \hline
   & predicted 1 & predicted 0 \\
  \hline
  actual 1 & true positive & false negative \\
  actual 0 & false positive & false negative \\
  \hline
\end{tabular}
\end{center}

The total number of cases is: true positives + true negatives + false negatives + false positives.

\subsection{Accuracy}
Accuracy is defined as: (true positives + true negatives) / total cases.
If the data is severely skewed, the accuracy metric is misleading. If 250 000 credit card transactions are non-fraudulent and there are 500 fraudulent, a model that always predicts a non-fraudulent transaction has an accuracy of 99.8%.

\subsection{Precision}
Precision is defined as: out of all points predicted to be 1 how many are actually 1? The formulation is: true positives / (true positives + false positives). This corresponds to the first column of the confusion matrix. A spam filter model needs to have a high precision. This is because we want to have as few false positives (emails mistakenly identified as spam) as possible.

\subsection{Recall}
Recall can be defined as: out of all points that actually are 1, how many are correctly predicted as 1? The formulation is: true positives / (true positives + false negatives). This corresponds to the first row of the confusion matrix. A medical model needs to have a high recall in order to catch as many sick (positives) patients as possible. This prioritizes labeling the 1s correctly. This is because if a patient is falsely diagnosed as sick, they could be sent for further tests and could have some discomfort but no sick patient will be sent home.

\subsection{F1 Score}
F1 score is defined as:

\begin{equation}
  F1 = 2 * \frac{precision * recall}{precision + recall}
\end{equation}

This is the harmonic mean of precision and recall.

\subsection{Fbeta Score}
Fbeta score is defined as:

\begin{equation}
  F_\beta = (1 + \beta^2) * \frac{precision * recall}{\beta^2 * precision + recall}
\end{equation}

Lower beta values skew the formula towards precision ($F_0 = precision$) and higher beta values skew the score towards recall ($F_\infty = recall$).

True positive rate is true positives / all positives, while the false positive rate is the false positives / all negatives.

\subsection{ROC Curves}
Receiver Operating Characteristic (ROC) curves. This is a curve created by plotting the true positive rate and false positive rate as the x, y coordinates of every possible split of the model. This gives us a 2 dimensional curve. The area under this curve is the ROC score. It is 1 for a perfect classifier and 0.5 for a random classifier. The ROC score could be 0 if the model misclassifies every single point.

\subsection{Regression Metrics}

\begin{itemize}
  \item mean absolute error
  \item mean squared error (MSE)
  \item R2 score
\end{itemize}

R2 score is a comparison between our model and the simplest possible model. For a linear regression, such a baseline model could just be a horizontal line drawn between the points. In this case, the R2 score is:

\begin{equation}
  R2 = 1 - \frac{MSE (current)}{MSE (baseline)}
\end{equation}

\section{Model Selection}
\subsection{K-Fold Cross Validation}
It’s method of recycling our data in order to better utilize it for training and testing. Using this method we don’t ‘lose’ data for testing without violating the rule of use of the test set (never use the test set for training).

\begin{enumerate}
  \item We break the data into K buckets.
  \item We train our model K times, each time using a different bucket as our testing set (and the remaining points as our training set).
  \item We average the results from the K training to get a final model.
\end{enumerate}

\subsection{Learning Curves}
If we plot on the x axis the number of training points used and on the y axis error, we can plot 2 series:

\begin{itemize}
  \item Error on training set
  \item Error on testing set
\end{itemize}

Those two curves behave differently depending on the goodness of the model:

\begin{itemize}
  \item High bias model (underfitting): the training and the testing errors converge to a medium to high error.
  \item Good model: the training and testing errors converge to a low error.
  \item High variance model (overfitting): the training and testing errors don’t converge. This is because overfitted models always have very low errors on the training set, but don’t generalize well, therefore having a large error on the testing set.
\end{itemize}

\subsection{Grid Search}
Grid search is a technique for hyperparameter tuning. It creates a n-dimensional grid of all hyperparameters and their possible values and tests every cell of the grid against a validation set. The best model is then chosen and tested on the testing set.

For hyperparameters, it’s generally recommended to take values that grow exponentially (e.g. 1, 10, 100, 1000, etc.).

\section{Supervised Learning}
\subsection{Linear Regression}
The linear regression algorithm draws a line that fits best a group of points. The line in two dimensions is determined by the linear equation:

\begin{equation}
  y = w_1 x + w_2
\end{equation}

Here, $w_1$ is the slope and $w_2$ is the intercept. Linear regression can be done using several different methods. Two of these methods are:

\begin{itemize}
  \item The Absolute Trick
  \begin{enumerate}
    \item We start with a line: $y = w_1 x + w_2$
    \item We have a point with coordinates $(p, q)$, where p is the horizontal coordinate.
    \item At each step, we translate and rotate the line according to the equation: $y = (w_1 + p * \alpha) * x + (w_2 + \alpha)$, where $\alpha$ is the learning rate. If the line is on top of the point, we subtract the learning rate instead of adding it. The horizontal coordinate P is used in the equation in order to account for the case where that coordinate is negative. In this case the line will rotate accordingly. Also, if p is small (close to the y axis), the rotation will be small.
    \item We repeat step 3 until the error is minimized.
  \end{enumerate}
  \item The Square Trick
  \begin{enumerate}
    \item We start with a line: $y = w_1 x + w_2$
    \item We have a point with coordinates (p, q), where p is the horizontal coordinate. We will designate q’ as the point where a line originating from (p, q) crosses the linear regression line. In this case q - q’ is the distance between the point and the linear regression line.
    \item At each step, we translate and rotate the line according to the equation: $y = (w_1 + p (q - q') * \alpha) * x + (w_2 + p (q - q') * \alpha)$. If the line is on top the distance will be negative so the line will translate and rotate down automatically. Using the distance as a multiplier to the learning rate helps the algorithm converge faster.
    \item We repeat step 3 until the error is minimized.
  \end{enumerate}
\end{itemize}

For the error, we can use the mean absolute error and the mean squared error. The mean absolute error is defined as:

\begin{equation}
  \frac{1}{m}\sum_{i = 1}^{m}|y - \hat{y}|
\end{equation}

The mean squared error is defined as:

\begin{equation}
  \frac{1}{m}\sum_{i = 1}^{m}(y - \hat{y})^2
\end{equation}

To minimize the error:

\begin{enumerate}
  \item We take the partial derivative of our chosen error function with regards to the weights of the line equation.
  \item We take a step towards lowering the derivative of the error.
  \item We repeat step 2 until we reach a minimum.
\end{enumerate}

In our case the derivative of the mean absolute error is the same as the equation from the ‘absolute trick’, while the derivative of the mean squared error is the same as the equation in the ‘square trick’.

\subsection{Batch VS Stochastic Gradient Descent}
In the context of linear regression, we could either:

\begin{itemize}
  \item Apply the absolute / square trick for every point of our data one by one and repeating it many times. When we do that, we get values to update our model weights with at every point. This approach is called stochastic gradient descent.
  \item Apply the absolute / square trick for all points of our data simultaneously. Here we add all the update values and update the weights at the end of the process. This approach is called batch gradient descent.
\end{itemize}

In practice (and it is applicable for large datasets), the dataset is split into small batches, gradient descent is run for every batch and the weights are updated. This is called mini-batch gradient descent.

\subsection{Regularization}
Used for selecting simpler models with higher validation error over complex models with small validation error to prevent overfitting. The general concept is that we add the complexity of the model to the error. One way to do is to add the absolute value of the non-intercept coefficients to the model. The simpler the model, the less coefficients there will be. Regularization can be done in two ways:

\begin{itemize}
  \item L1 regularization: We add the sum of the absolute values of the coefficients to the error of the model.
  \item L2 regularization: We add the squares of the coefficients.
\end{itemize}

Some models need to be less complex (e.g. video recommendation system that works with a lot of data and performance is important), while other models need to be more precise, no matter the complexity (e.g. medical model). That’s why we can use a parameter lambda (ƛ) to determine the level of complexity punishment. Small lambda allow for more complex models, while large lambda increases the complexity punishment.

Here is a L1, L2 regularization cheatsheet:

\begin{center}
\begin{tabular}{ |l|l|l| }
  \hline
   Feature & L1 & L2 \\
  \hline
  computationally efficient & No & Yes \\
  faster for sparse data & Yes & No \\
  \makecell{detects important features and sets the \\ punishment for unimportant ones to 0} & Yes & No \\
  \hline
\end{tabular}
\end{center}

\subsection{Perceptron}
In a 2-dimensional classification problem, we need to split the space of two features into two regions (corresponding to the two classification labels). Let’s say that we can split the two regions with a simple line, where it’s equation is: $w_1 x_1 + w_2 x_2 + b = 0$. We can express this as Wx + b = 0, where W is the vector of the weights (w1, w2) and x is the vector of the coordinates of a given point (x1, x2). We can then say that our classification function $\hat{y}$ is:

\begin{equation}
  \hat{y} = \left\{
              \begin{array}{ll}
                1 , Wx + b \geq 0\\
                0 , Wx + b < 0
                \end{array}
              \right.
\end{equation}

Defined such, this technique can be extended to higher dimensions. There, the vectors W and x are simply defined as ($w_1$, $w_2$, …, $w_n$) and ($x_1$, $x_2$, …, $x_n$).

In order to update the weights of the perceptron, for every misclassified point we add or subtract (depending on the classification label) the points coordinates times the learning rate to the weights. This slightly changes the equation of the line (plane, hyperplane, etc.). We repeat this process until the decision boundary is optimal with as little misclassifications as possible.

\subsection{Decision Trees}
The more knowledge we have of a system, the less informational entropy it has. We will define informational entropy as:

\begin{equation}
  E = - \sum_{i = 1}^{n} p_i \log_2(p_i)
\end{equation}

Here $p_i$ is the probability of the $i^{th}$ feature.

Information gain is defined as change in entropy:

\begin{equation}
  InfGain = E(parent) - \left( \frac{m}{m + n} E(child_1) + \frac{m}{m + n} E(child_2) \right)
\end{equation}

Here $m$ and $n$ are the number of samples in each node.

This means that for each split, our decision tree needs to maximize information gain. This means that the tree will make random splits and measure the information gain. At each step the tree tries to make a new split that maximizes the information gain. When no more improvement is done, the tree attempts a new split. The problem with that approach is that it tends to overfit a lot.

One way to solve this problem is to pick few random sets of only 2-3 features, build trees on them and let them vote. The average prediction of all small trees is the prediction of the ensemble. This approach is known as random forests.

\end{document}
