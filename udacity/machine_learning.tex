\documentclass{article}
\usepackage{amsmath}
\usepackage{mathtools}
\usepackage{framed} %for important notes that need to be framed
\usepackage{tikz}
\usepackage{multicol} %for multi-column bullet points
\usepackage{float} %for placing figures at the exact place as in the code
\usepackage{lscape} %for inserting landscape pages
\usepackage{makecell} %for specifying line breaks within table cells

\usepackage[a4paper, total={7in, 10in}]{geometry} %for formatting the sheet size

\usepackage{graphicx}
\graphicspath{ {./visualizations/} } %for inserting figures

\usepackage{hyperref} %for hyperlinks
\hypersetup{
    colorlinks = true,
    linkcolor = blue,
    filecolor = magenta,
    urlcolor = cyan,
}

\title{Udacity Machine Learning Nanodegree - Course Notes}

\author{Ruslan Kozhuharov}

\begin{document}
\pagenumbering{gobble}
\maketitle
\tableofcontents
\newpage
\pagenumbering{arabic}
\setlength{\parskip}{1em}

\section{Model Evaluation}
\subsection{Confusion Matrix}
The following table is called a confusion matrix:

\begin{center}
\begin{tabular}{ |l|l|l| }
  \hline
   & predicted 1 & predicted 0 \\
  \hline
  actual 1 & true positive & false negative \\
  actual 0 & false positive & false negative \\
  \hline
\end{tabular}
\end{center}

The total number of cases is: true positives + true negatives + false negatives + false positives.

\subsection{Accuracy}
Accuracy is defined as: (true positives + true negatives) / total cases.
If the data is severely skewed, the accuracy metric is misleading. If 250 000 credit card transactions are non-fraudulent and there are 500 fraudulent, a model that always predicts a non-fraudulent transaction has an accuracy of 99.8%.

\subsection{Precision}
Precision is defined as: out of all points predicted to be 1 how many are actually 1? The formulation is: true positives / (true positives + false positives). This corresponds to the first column of the confusion matrix. A spam filter model needs to have a high precision. This is because we want to have as few false positives (emails mistakenly identified as spam) as possible.

\subsection{Recall}
Recall can be defined as: out of all points that actually are 1, how many are correctly predicted as 1? The formulation is: true positives / (true positives + false negatives). This corresponds to the first row of the confusion matrix. A medical model needs to have a high recall in order to catch as many sick (positives) patients as possible. This prioritizes labeling the 1s correctly. This is because if a patient is falsely diagnosed as sick, they could be sent for further tests and could have some discomfort but no sick patient will be sent home.

\subsection{F1 Score}
F1 score is defined as:

\begin{equation}
  F1 = 2 * \frac{precision * recall}{precision + recall}
\end{equation}

This is the harmonic mean of precision and recall.

\subsection{Fbeta Score}
Fbeta score is defined as:

\begin{equation}
  F_\beta = (1 + \beta^2) * \frac{precision * recall}{\beta^2 * precision + recall}
\end{equation}

Lower beta values skew the formula towards precision ($F_0 = precision$) and higher beta values skew the score towards recall ($F_\infty = recall$).

True positive rate is true positives / all positives, while the false positive rate is the false positives / all negatives.

\subsection{ROC Curves}
Receiver Operating Characteristic (ROC) curves. This is a curve created by plotting the true positive rate and false positive rate as the x, y coordinates of every possible split of the model. This gives us a 2 dimensional curve. The area under this curve is the ROC score. It is 1 for a perfect classifier and 0.5 for a random classifier. The ROC score could be 0 if the model misclassifies every single point.

\subsection{Regression Metrics}

\begin{itemize}
  \item mean absolute error
  \item mean squared error (MSE)
  \item R2 score
\end{itemize}

R2 score is a comparison between our model and the simplest possible model. For a linear regression, such a baseline model could just be a horizontal line drawn between the points. In this case, the R2 score is:

\begin{equation}
  R2 = 1 - \frac{MSE (current)}{MSE (baseline)}
\end{equation}

\section{Model Selection}
\subsection{K-Fold Cross Validation}
It’s method of recycling our data in order to better utilize it for training and testing. Using this method we don’t ‘lose’ data for testing without violating the rule of use of the test set (never use the test set for training).

\begin{enumerate}
  \item We break the data into K buckets.
  \item We train our model K times, each time using a different bucket as our testing set (and the remaining points as our training set).
  \item We average the results from the K training to get a final model.
\end{enumerate}

\subsection{Learning Curves}
If we plot on the x axis the number of training points used and on the y axis error, we can plot 2 series:

\begin{itemize}
  \item Error on training set
  \item Error on testing set
\end{itemize}

Those two curves behave differently depending on the goodness of the model:

\begin{itemize}
  \item High bias model (underfitting): the training and the testing errors converge to a medium to high error.
  \item Good model: the training and testing errors converge to a low error.
  \item High variance model (overfitting): the training and testing errors don’t converge. This is because overfitted models always have very low errors on the training set, but don’t generalize well, therefore having a large error on the testing set.
\end{itemize}

\subsection{Grid Search}
Grid search is a technique for hyperparameter tuning. It creates a n-dimensional grid of all hyperparameters and their possible values and tests every cell of the grid against a validation set. The best model is then chosen and tested on the testing set.

For hyperparameters, it’s generally recommended to take values that grow exponentially (e.g. 1, 10, 100, 1000, etc.).

\section{Supervised Learning}
\subsection{Linear Regression}
The linear regression algorithm draws a line that fits best a group of points. The line in two dimensions is determined by the linear equation:

\begin{equation}
  y = w_1 x + w_2
\end{equation}

Here, $w_1$ is the slope and $w_2$ is the intercept. Linear regression can be done using several different methods. Two of these methods are:

\begin{itemize}
  \item The Absolute Trick
  \begin{enumerate}
    \item We start with a line: $y = w_1 x + w_2$
    \item We have a point with coordinates $(p, q)$, where p is the horizontal coordinate.
    \item At each step, we translate and rotate the line according to the equation: $y = (w_1 + p * \alpha) * x + (w_2 + \alpha)$, where $\alpha$ is the learning rate. If the line is on top of the point, we subtract the learning rate instead of adding it. The horizontal coordinate P is used in the equation in order to account for the case where that coordinate is negative. In this case the line will rotate accordingly. Also, if p is small (close to the y axis), the rotation will be small.
    \item We repeat step 3 until the error is minimized.
  \end{enumerate}
  \item The Square Trick
  \begin{enumerate}
    \item We start with a line: $y = w_1 x + w_2$
    \item We have a point with coordinates (p, q), where p is the horizontal coordinate. We will designate q’ as the point where a line originating from (p, q) crosses the linear regression line. In this case q - q’ is the distance between the point and the linear regression line.
    \item At each step, we translate and rotate the line according to the equation: $y = (w_1 + p (q - q') * \alpha) * x + (w_2 + p (q - q') * \alpha)$. If the line is on top the distance will be negative so the line will translate and rotate down automatically. Using the distance as a multiplier to the learning rate helps the algorithm converge faster.
    \item We repeat step 3 until the error is minimized.
  \end{enumerate}
\end{itemize}

For the error, we can use the mean absolute error and the mean squared error. The mean absolute error is defined as:

\begin{equation}
  \frac{1}{m}\sum_{i = 1}^{m}|y - \hat{y}|
\end{equation}

The mean squared error is defined as:

\begin{equation}
  \frac{1}{m}\sum_{i = 1}^{m}(y - \hat{y})^2
\end{equation}

To minimize the error:

\begin{enumerate}
  \item We take the partial derivative of our chosen error function with regards to the weights of the line equation.
  \item We take a step towards lowering the derivative of the error.
  \item We repeat step 2 until we reach a minimum.
\end{enumerate}

In our case the derivative of the mean absolute error is the same as the equation from the ‘absolute trick’, while the derivative of the mean squared error is the same as the equation in the ‘square trick’.

\subsection{Batch VS Stochastic Gradient Descent}
In the context of linear regression, we could either:

\begin{itemize}
  \item Apply the absolute / square trick for every point of our data one by one and repeating it many times. When we do that, we get values to update our model weights with at every point. This approach is called stochastic gradient descent.
  \item Apply the absolute / square trick for all points of our data simultaneously. Here we add all the update values and update the weights at the end of the process. This approach is called batch gradient descent.
\end{itemize}

In practice (and it is applicable for large datasets), the dataset is split into small batches, gradient descent is run for every batch and the weights are updated. This is called mini-batch gradient descent.

\subsection{Regularization}
Used for selecting simpler models with higher validation error over complex models with small validation error to prevent overfitting. The general concept is that we add the complexity of the model to the error. One way to do is to add the absolute value of the non-intercept coefficients to the model. The simpler the model, the less coefficients there will be. Regularization can be done in two ways:

\begin{itemize}
  \item L1 regularization: We add the sum of the absolute values of the coefficients to the error of the model.
  \item L2 regularization: We add the squares of the coefficients.
\end{itemize}

Some models need to be less complex (e.g. video recommendation system that works with a lot of data and performance is important), while other models need to be more precise, no matter the complexity (e.g. medical model). That’s why we can use a parameter lambda (ƛ) to determine the level of complexity punishment. Small lambda allow for more complex models, while large lambda increases the complexity punishment.

Here is a L1, L2 regularization cheatsheet:

\begin{center}
\begin{tabular}{ |l|l|l| }
  \hline
   Feature & L1 & L2 \\
  \hline
  computationally efficient & No & Yes \\
  faster for sparse data & Yes & No \\
  \makecell{detects important features and sets the \\ punishment for unimportant ones to 0} & Yes & No \\
  \hline
\end{tabular}
\end{center}

\subsection{Perceptron}
In a 2-dimensional classification problem, we need to split the space of two features into two regions (corresponding to the two classification labels). Let’s say that we can split the two regions with a simple line, where it’s equation is: $w_1 x_1 + w_2 x_2 + b = 0$. We can express this as Wx + b = 0, where W is the vector of the weights (w1, w2) and x is the vector of the coordinates of a given point (x1, x2). We can then say that our classification function $\hat{y}$ is:

\begin{equation}
  \hat{y} = \left\{
              \begin{array}{ll}
                1 , Wx + b \geq 0\\
                0 , Wx + b < 0
                \end{array}
              \right.
\end{equation}

Defined such, this technique can be extended to higher dimensions. There, the vectors W and x are simply defined as ($w_1$, $w_2$, …, $w_n$) and ($x_1$, $x_2$, …, $x_n$).

In order to update the weights of the perceptron, for every misclassified point we add or subtract (depending on the classification label) the points coordinates times the learning rate to the weights. This slightly changes the equation of the line (plane, hyperplane, etc.). We repeat this process until the decision boundary is optimal with as little misclassifications as possible.

\subsection{Decision Trees}
The more knowledge we have of a system, the less informational entropy it has. We will define informational entropy as:

\begin{equation}
  E = - \sum_{i = 1}^{n} p_i \log_2(p_i)
\end{equation}

Here $p_i$ is the probability of the $i^{th}$ feature.

Information gain is defined as change in entropy:

\begin{equation}
  InfGain = E(parent) - \left( \frac{m}{m + n} E(child_1) + \frac{m}{m + n} E(child_2) \right)
\end{equation}

Here $m$ and $n$ are the number of samples in each node.

This means that for each split, our decision tree needs to maximize information gain. This means that the tree will make random splits and measure the information gain. At each step the tree tries to make a new split that maximizes the information gain. When no more improvement is done, the tree attempts a new split. The problem with that approach is that it tends to overfit a lot.

One way to solve this problem is to pick few random sets of only 2-3 features, build trees on them and let them vote. The average prediction of all small trees is the prediction of the ensemble. This approach is known as random forests.

\subsection{Naive Bayes}
The Bayes theorem switches from what we know to what we can infer. We take what we know and base what we can infer on it. Example: if we know that when it rains, the weather is cloudy, then we try to infer whether it will rain based on the fact that it is cloudy. If we know the probability of R given A, we infer the probability of A given R: $P(R|A) \rightarrow P(A|R)$.

The simple version of the Bayes theorem is:

\begin{equation}
  P(A|B) = \frac{P(A) P(B|A)}{P(B)}
\end{equation}

The ‘naive’ in the Naive Bayes algorithm is the assumption that all events, for which we calculate the posterior probability are independent of each other. That means that we can multiply the posterior probabilities of all the possible events for class 1 and class 0. We then multiply them by the probability of class 1 or 0. Finally, we normalize:

\begin{equation}
  \begin{aligned}
    P(A_1|B, C, D) &\sim P(B|A_1) P(C|A_1) P(D|A_1) P(A_1) \\
    P(A_2|B, C, D) &\sim P(B|A_2) P(C|A_2) P(D|A_2) P(A_2) \\
    P(A_1|B, C, D) &= \frac{P(A_1|B, C, D)}{P(A_1|B, C, D) + P(A_2|B, C, D)}
  \end{aligned}
\end{equation}

\subsection{Support Vector Machines}
SVM is an algorithm that tries to find a classification boundary in a set of points where the margin between the two classes is as wide as possible. This is achieved by minimizing the sum of the classification error and the margin error.

The classification error is the sum of the absolute values of the line $Wx + b$ at every misclassified point (where W and x are vectors). This error resembles the perceptron function error.

\begin{framed}
  Example: if a misclassified point is on a line parallel to the perfect split and the distance between the perfect line and the misclassification line is 1, then the point lies on the line: $Wx + b = 1$.
\end{framed}

In order to punish the points in the boundary for the SVM algorithm, we consider the error of one class starting from: $Wx + b = 1$ and for the other class from: $Wx + b = -1$. That way we are punishing the misclassified points as well as the points in the boundary.

If we have two support vectors with equations $Wx + b = 1$ and $Wx + b = -1$, then the margin is defined as:

\begin{equation}
  M = \frac{2}{|W|}
\end{equation}

So if we want to make the error inversely-proportional to the margin, we can just take the norm of the weights vector: $|W|$. The margin error is then defined as the norm of the components of vector $W$ squared:

\begin{equation}
  E = |W|^2
\end{equation}

The final error function is the classification error plus the margin error. We minimize this error using gradient descent.

The ‘C’ parameter is multiplier to the classification error, that way we have the flexibility to specify in any model, how much the classification error and how much the margin error matters:

\begin{equation}
  E = C * E_{classification} + E_{margin}
\end{equation}

\subsubsection{Polynomial Kernel}
A polynomial kernel takes the data to a higher dimension, creates the corresponding polynomial, extracts the coefficients and returns them as solutions to the lower dimensions. One example is if we have a line with points that are inseparable by a single cut. We then arrange them on the parabolic function $y = x^2$.

We place the points on the parabola. If we find a cut in the line $y = 4$, we translate the solution to 1 dimension. We do that by solving for the polynomial (the solutions are 2 and -2). So as a result we have two cuts that separate the points. This is also known as the kernel trick. Thus, if we want to raise a 1 dimensional data to a higher dimension, we could use a function of the type: $x^2$, if we want to raise a 2 dimensional data to a higher dimension, we can use: $xy, x^2y, x^3$, etc.

The degree of a polynomial kernel is a hyperparameter that can be trained to find the best possible model.

\subsubsection{Radial Basis Function Kernel}
In this method, we draw the radial basis functions for every point. We then iterate through radius vectors that are suitable for cutting the different classes of points with an object in the same dimension (line, circle, plane, etc.). Finally, we project down the object to the original dimensions. For 1-dimensional data, the result from projecting another line are the points where we cut the classes. The hyper-parameter gamma allows us to specify the curve radius for every radial basis function. Large gamma gives us ‘pointy mountains’, while small one makes the curves rounder. This is because gamma is defined (in terms of the normal distribution) as: $\gamma = \frac{1}{2\sigma^2}$. Large gammas translate as very narrow class decision areas and therefore have a high chance of overfitting.

\subsection{Ensemble Methods}
Ensemble methods are a set of techniques where we utilize several weak models to create a strong prediction. In general, there are two categories of ensemble methods: bagging (bootstrap aggregating) and boosting.

\begin{itemize}
  \item Bagging is a method where we get predictions from the weak learners, we then average the prediction and get our result.
  \item Boosting is a method where each weak learner focuses only on a small sample of the dataset and attempts to predict it correctly. We then combine the predictions for different areas of the dataset from different weak learners to create a strong learner.
\end{itemize}

\subsubsection{Bagging Method}
With the bagging method, we usually utilize tree stubs (decision trees with only 1 level) on random subsets of the data. This means that two samples can pick the same point (samples can overlap) and this is OK as long as the samples are random. For each random sample, we create a stub. At the end we combine the results of the stubs by voting. Example: a point is covered by 3 stubs, 2 of them vote category 1 and the other one votes category 0, therefore, we predict category 1.

\subsubsection{Boosting Method With AdaBoost}
We’ll attempt to minimize the sum of the weights of the incorrectly-classified points.

\begin{enumerate}
  \item We assign to each point in the dataset an initial weight of 1.
  \item We create a decision tree stub.
  \item We change the weights of the incorrectly-classified points to: number correctly-classified / number incorrectly-classified (from the stub in step 2).
  \item We repeat step 2-3 for certain number of cycles.
  \item We assign a weight to all the weak learners produced in step 4. The weight is: $w = ln \left( \frac{accuracy}{1 - accuracy} \right)$. In the extreme cases of division by zero or accuracy 0 (where $ln(0)$ is undefined), we know that the model either predicts the data infinitely-correctly or infinitely-incorrectly, so we can ignore all the other stubs and just use the current one. This case, however is almost impossible to happen in practice.
  \item We combine the stubs from step 4 by vote and weighting the vote by the weights from step 5. This vote is basically the weighted sum of the model predictions for every point. If the values are positive, the result is class 1, if the values are negative, the result is class 0.
\end{enumerate}

\section{Unsupervised Learning}
\subsection{K-Means Clustering}
K-means clustering assigns a user-specified number of clusters to unlabeled data. There are two steps in the K-means algorithm:

\begin{itemize}
  \item Assignment: where points are assigned to their centroid.
  \item Optimization: where the centroid is moved to an optimal location.
\end{itemize}

The essence of the algorithm is to select cluster centroids such that minimize the within-cluster sum of squares, expressed in the following equation:

\begin{equation}
  \sum_{i = 1}^{n} min_{\mu j \in C} \left( ||x_j - \mu_j||^2 \right)
\end{equation}

K-means is very dependent on the initial placements of the centroids. In this case, the danger is the so-called ‘unsupervised learning local minimum’ where 3 neat clusters could be misrepresented as 1 big clusters (containing 2 groups of data points) and 2 small clusters (dividing evenly 1 actual group of data points).

The K-means clustering algorithm is advantageous if we know the number of clusters beforehand and if they are neatly-separated. However, K-means relies on distance between points and centroids. Therefore, the results of applying of K-means are always circular, spherical or hyperspherical. K-means is not able to carve out the two-crescent dataset, for example.

\subsection{Hierarchical Clustering}
Hierarchical clustering methods assume that initially every point is a cluster. At the next pass, points closest to each other form a cluster of two points. The constitutes one level of the hierarchical tree (dendrogram). At this point the hierarchical clustering methods start combining points with existing clusters (from the first pass). There lies the principal differences between all hierarchical clustering methods - in the way the smallest distance between a cluster and a point is measured. In other words, the specific clustering method answers the question: how is the shortest distance between two clusters defined? Is it the shortest distance between points from each cluster or is it some averaged variable.




\end{document}
