\documentclass{article}
\usepackage{amsmath}
\usepackage{framed}

\title{Udacity Reinforcement Learning Course Notes}
\author{Ross K}

\begin{document}
\pagenumbering{gobble}
\maketitle
\tableofcontents
\newpage
\pagenumbering{arabic}

\section{Introduction}
The main types of learning are:
  \begin{itemize}
    \item Supervised or function approximation
    \item Unsupervised or clustering description
    \item Reinforcement or production of both function and results
  \end{itemize}
\section{Markov Model}
List of important terms:
  \begin{itemize}
    \item States: all configurations the system could be in
    \item Actions: what can be done in a particular state
    \item Markovian property: the probability of ending up in a state S’ only depends on the current state S
    \item Rewards: a scalar value one gets for being in a state
    \item Policy: solution to a Markov model
  \end{itemize}
The Markov policies are robust to the underlying stochasticity of the world. Unlike planning, that tells you a sequence of actions and some of the actions can have unintended consequences. The Markov policies instead tell you what action should be taken in any state.
\subsection{Credit Assignment Problem}
In the Markov Decision Process, every action leads to a new state. The new state may lead to new actions and so forth until the final state. Only the final state assigns a reward. The credit assignment problem is the one of figuring out what sequence of actions would lead to a reward in the long run. In contrast, in supervised learning, every action has a reward, so we are trying to build a function that maximizes reward for every action.
\subsection{Sequences of Rewards}
Utility for infinite rewards:
\begin{equation*}
  U(S_0, S_1, S_2, ...) = \sum_{t=0}^{\infty} R(S_t) = \infty
\end{equation*}
In this construct, the rewards (R) from all states (S) accumulate and always reach infinity.
Utility for finite rewards (discounted rewards):
\begin{equation*}
  U(S_0, S_1, S_2, ...) = \sum_{t=0}^{\infty} \gamma^t R(S_t) = \infty, 0 <= \gamma < 1
\end{equation*}
In this construct, the rewards get discounted, so each following reward is worth progressively less than the previous one. This means that infinite amount of rewards can add up to a finite reward sum. This finite amount can be expressed as: $\frac{R_{max}}{1-\gamma}$
Optimal policy is defined as:
\begin{equation*}
  \pi^* = argmax_{\pi} \exists \left[ \sum_{t=0}^{\infty} \gamma^t R(S_t) | \pi \right]
\end{equation*}
The above means that the optimal policy is the one that maximizes the sum of the discounted rewards.
The utility of a particular state depends upon the policy we are following. And that is the expected set of states we are going to see from that point on given that we follow a policy (i.e. if that state is S0). This can be expressed as:
\begin{equation*}
  U^{\pi}(s) = \exists \left[ \sum_{t=0}^{\infty} \gamma^t R(S_t) | \pi, S_0 = S \right]
\end{equation*}
\begin{framed}
Important: The reward for entering a state is not the utility of that state. The reward gives us the immediate feedback, but the utility gives is the long term feedback.
\end{framed}
The optimal action to take at a state is look at all the actions and sum up all the next states, the transition probabilities (the probability of ending in $S'$) times the utility of $S'$:
\begin{equation*}
  \pi^*(S) = argmax_a \sum_{S'} T(S, a, S') U(S')
\end{equation*}
The true utility of a state is the utility where the optimal policy is followed:
\begin{equation*}
  U(S) \equiv U^{\pi^*}(s)
\end{equation*}
The utility of a state is the reward for entering the state plus the discounted rewards we are going to get from that point on (the Bellmont Equation):
\begin{equation}
  U(S) = R(S) + \gamma_{max_a} \sum_{S'} T(S, a, S') U(S')
\end{equation}

\end{document}
